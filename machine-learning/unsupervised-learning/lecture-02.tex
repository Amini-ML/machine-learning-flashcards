%Front;
% Probabilistic and Unsupervised Learning
% Lectured by Maneesh Sahani at the Gatsby Computational Neuroscience Unit, UCL, Fall 2018
% Link to course info and slides: http://www.gatsby.ucl.ac.uk/teaching/courses/ml1/
\documentclass{article}
\usepackage{amsmath}
\begin{document}
	
\section{Distributions}

% beta already included in lecture 1

Dirichlet distribution; $p(\mathbf{x}|\mathbf{\alpha}) = \frac{\Gamma (\sum_{d=1}^{D}\alpha_d)}{\prod_{d=1}^{D}\Gamma(\alpha_d)}\prod_{d=1}^{D}x_d^{\alpha_{d}-1}$, $\mathbf{x}\in [0,1]^D$, $\mathbf{x}\sim Dirichlet(\alpha)$

Gamma distribution; $p(x|\alpha, \beta) = \frac{\beta^{\alpha}}{\Gamma(\alpha)}x^{\alpha-1}e^{-\beta x}$, $x\in \Re_+$

Poisson distribution; $p(x|\mu) = \frac{\mu^xe^{-\mu}}{x!}$, $x\in\mathcal{Z}_{0+}$

Multinomial distribution; $p(\mathbf{x|p})=\frac{N!}{x_1!x_2!...x_D!}\prod_{d=1}^{D}p_d^{x_d}$, $\mathbf{x}\sim [\mathcal{Z}_{0-N}]^D$

Binomial distribution; $p(x|p) = {N\choose x} p^x(1-p)^{(N-x)}$

Multivariate Gaussian distribution; $p(\mathbf{x}|\mathbf{\mu, \Sigma}) = |2\pi\Sigma|^{-1/2}\exp(-\frac{1}{2}(\mathbf{x-\mu})^T\Sigma^{-1}(\mathbf{x-\mu})$

What is one way of making generative models that are more expressive than those in the exponential family (but still involves using exponential family dists)?; Combining and transforming latent exponential family variates.

What is the structure of a simple two-variable latent variable model?; \begin{itemize} \item Suppose $y~P[\theta_y]$, $x|y~P[\theta_x]$.  \item $p(x,y|\theta_x, \theta_y) = p(x|y, \theta_x)p(y|\theta_y)$ \item $p(x|\theta_x, \theta_y) = \int dy p(x|y, \theta_x) p(y|\theta_y)$.  \end{itemize}

Conditional independence relationship between x, $\theta_y$ in latent variable model; Given $y$, $x$ is conditionally independent of $\theta_y$.

Likelihood vs likelihood over latents; \begin{itemize}	\item Likelihood: $p(x|\theta)$,	\item Likelihood over latents: $p(x|y, \theta)$ \end{itemize}

Why might we use latent variable models?; \begin{itemize} \item Describe structured distributions \begin{itemize} \item correlations in high-dim x may be captured by fewer parameters.  \end{itemize} \item Capture an underlying generative process \begin{itemize} \item y may describe causes of x.  \item help to separate signal from noise.  \end{itemize} \item Combine exponential family dists into richer, more flexible forms.  \begin{itemize} \item P(y), $P(x|y)$ and even P(x,y) may be in the exponential family \item P(x) rarely is. (Exception: linear Gaussian models) \end{itemize} \end{itemize}

If $P(y)$ and $P(x|y)$ live in the exponential family, does $P(x,y)$ live in the exponential family?; Not necessarily - x-y coupling in product may mean this doesn't live in the exp family.

How can we decompose a Gaussian with correlated noise, e.g. $x~N(0, [ 3 2 / 2 3])$?; $y~N(0,1)$, $x~N(\sqrt{2}[1 / 1]y, I)$.

Isotropic (noise); scalar*identity matrix.

Required relationship between K and D for factor analysis; K $<<$ D for model to be identifiable

Is trace a linear operator?; Yes. $Tr(A+B) = Tr(A)+Tr(B)$. $kTr(A) = Tr(kA)$ for scalar $k$.

If two or more eigenvectors share an eigenvalue $\lambda$, what can we say about the eigenvectors?; they are degenerate: any linear combination $\alpha u_1 + (1-\alpha)u_2$ is also an eigenvector.

PPCA model: as $\psi\rightarrow 0$, how many dimensions of variance can the model capture?; K dims

Describe the probabilistic PCA model.; \begin{itemize} \item N D-dim datapoints $\mathbf{x_i} \in \mathcal{R}^D$ \item N k-dimensional latents $y_i \in \mathcal{R}^K$ \item Linear generative model: $x_d = \sum_{k=1}^k \Lambda_{dk}y_k + \epsilon_d$, where \begin{itemize} \item $y_k$ are independent $N(0,1)$ Gaussian factors \item $\epsilon_d$ are independent $N(0,\psi)$ Gaussian noise \item latent dim K $<$ D (obs dim) \item $\Lambda$ is a D x K matrix.  \end{itemize} \item $p(x|y) = N(\Lambda y,\phi I)$ \item $p(x) = \int p(y)p(x|y)dy = N(E_y[\Lambda y], E_y[\Lambda y y^T\Lambda^T]+\psi I)/N(0,\Lambda\Lambda^T+\psi I)$ \end{itemize}

Law of iterated expectation; $E_x[f(x)] = E_y[E_{x|y}[f(x)]]$

Variance of x in terms of expectation, variance over y, where x can be conditioned on y; \begin{itemize}
    \item $Var_X[x] = E_Y[V[x|y] + V[E[x|y]]]$
    \item Unexplained + explained
    \item Write out defn, use law of iterated expectations, rewrite $E[X^2|Y]=Var[X]+E[X|Y]^2$, regroup terms.
    \item called the `Law of total variance'.
\end{itemize} 

Compare modelling x as a multivariate Gaussian vs using probabilistic PCA; \begin{itemize}\item Multivariate Gaussian \begin{itemize} \item Model: Descriptive density model: correlations captured by off-diagonal elements of $\Sigma$.  \item $\Sigma$ has $\frac{D(D+1)}{2}$ free parameters.  \item $\Sigma$ only constrained to be positive definite.  \item Simple ML estimate.  \end{itemize} \item Latent variables \begin{itemize} \item Interpretable causal model: correlations caputerd by common influence of latent variable.  \item $\Lambda\Lambda^T +\psi I$ has $DK+1$ free parameters.  \item For $K < D$, covariance structure (of x?) is constrained (blurry pancake) \item ML estimation is more complex.  \end{itemize} \end{itemize}

Probabilistic PCA log likelihood (over entire dataset); $\log p(\chi | \Lambda, \psi) = -\frac{N}{2}\log|2\pi(\Lambda\Lambda^T+\psi I)|-\frac{1}{2}Tr[(\Lambda\Lambda^T+\psi I)^{-1}\sum_n\mathbf{x x^T}]$

[TODO: how to do maximum likelihood over PPCA likelihood for $(\Lambda, \psi)$?] (can do gradient ascent, but there is a 'simpler principle')

Link between probabilistic PCA and PCA; PPCA $\to$ PCA when $\psi\to 0$, where $\psi$ is the magnitude of the additive noise to x conditional on y.

Describe PCA; Assume data have zero mean. \begin{itemize} \item Find direction of greatest variance $\mathbf{\lambda}_{(1)}=\arg\max\limits_{||\mathbf{v}||=1}\sum_n(\mathbf{x^T_nv})^2$ \item Find direction orthogonal to $\mathbf{\lambda}_{n}$ with greatest variance, call this  $\mathbf{\lambda}_{n+1}$.  \item Terminate when remaining variance drops below a threshold.  \item These turn out to be the eigenvectors of the empirical covariance matrix $S=\langle \mathbf{xx^T} \rangle$ \end{itemize}

Express in algebraic terms the fact that the D eigenvectors form an orthonormal basis.; $\sum_i\mathbf{u_{i}u_{i}^T}=I$.

Rewrite a vector $v$ in terms of the orthonormal basis vectors $u$.; \begin{itemize} \item $\mathbf{v}=(\sum_i \mathbf{u_{(i)}u_{(i)}^T})\mathbf{v}$ \item $=\sum_i (\mathbf{u_{(i)}^Tv)u_{(i)}}$ \item $=\sum_i v_{(i)}\mathbf{u_{(i)}}$ \end{itemize}

Rewrite the empirical covariance matrix $S=\langle \mathbf{xx^T} \rangle$ in terms of the orthogonal basis vectors $\mathbf{u}_i$.; $S=\sum_i \omega_{(i)}\mathbf{u}_{(i)}\mathbf{u}_{(i)}^T = UWU^T$, where \begin{itemize} \item $U=[\mathbf{u_{1}}...]$ collects the eigenvectors and \item $W = diag[(\omega_{(1)}, \omega_{(2)},...,\omega_{(D)})]$.  \end{itemize}

Write in matrix form for all eigenvectors $Su = \omega u$.; $SU=UW$.

Given the direction of greatest variance is the eigenvector of the empirical covariance matrix S with the largest eigenvalue, show that the PCs are exactly the eigenvectors of S ordered by decreasing eigenvalue; TODO

How many eigenvalue-eigenvector pairs does the covariance matrix $S_{DXD}$ have?; \begin{itemize} \item Usually D pairs, \item except if two or more eigenvectors share the same eigenvalue (in which case the eigenvectors are degenerate - any linear combination is also an eigenvector).  \end{itemize}

Variance of empirical data in direction $\mathbf{u}_{(i)}$ (PCA); \begin{itemize} \item $\langle (\mathbf{x^Tu_{(i)}})^2\rangle $ \item $= \langle \mathbf{u_{(i)}^Txx^Tu_{(i)}}\rangle $ \item $=\mathbf{u_{(i)}^TSu_{(i)}}$ \item $=\mathbf{u_{(i)}^T\omega_{(i)}u_{(i)}}$ \item $=\omega_{(i)}$ \end{itemize}

Variance of empirical data in an arbitrary direction $\mathbf{v}$; \begin{itemize} \item $\langle (\mathbf{x^Tv})^2\rangle $ \item $= \langle (\mathbf{x^T(\sum_i v_{(i)}\mathbf{u}_{(i)})})^2\rangle $ \item $=\sum_{ij}v_{(i)}\mathbf{u_{(i)}^TSu_{(j)}}v_{(j)}$ \item $=\sum_{ij}v_{(i)}\omega_{(i)}v_{(j)}\mathbf{u_{(i)}^Tu_{(j)}}v_{(j)}$ \item $=\sum_i v_{(i)}^2\omega_{(i)}$ \end{itemize}

Explain how the PCs can be obtained from the empirical data.; \begin{itemize} \item PCs are the eigenvectors of the empirical covariance matrix $S=\langle\mathbf{xx^T}\rangle$ with the largest eigenvalues.  \item Variance in arbitrary direction is $\langle (\mathbf{x^Tv})^2\rangle =\sum_i v_{(i)}^2\omega_{(i)}$.  \item If $\mathbf{v^Tv}=1$, then $\sum_i v_{(i)}^2=1$ and so $\arg\max_{||v||=1} \langle (\mathbf{x^Tv})^2\rangle = \mathbf{u}_{(max)}$. ($v_{(i)}=1$ corresponding to largest eigenvalue, zero otherwise.) \item i.e. the direction of greatest variance is the eigenvector with the largest eigenvalue.  \end{itemize}

State the algebraic expression for a projection into a K-dimensional principal subspace.; $\hat{x_n} = \sum_{k=1}^K x_{n(k)}\mathbf{\lambda}_{(k)}$

State three equivalent definitions of PCA.; \begin{itemize} \item Find K directions of greatest variance in data.  \item Find K-dimensional orthogonal projection that preserves greatest variance.  \item Find K-dimensional vectors $y_i$ and matrix $\Lambda$ so that $\hat{\mathbf{x}}_i = \Lambda\mathbf{y_i}$ is as close as possible (in terms of squared distance) to $\mathbf{x_i}$.  \end{itemize}

Describe PCA from the perspective of mutual information.; \begin{itemize} \item Problem: Given $\mathbf{x}$, find $\mathbf{y = Ax}$ with columns of $A$ being unit vectors such that $I(y,x)$ is maximised, assuming $p(\mathbf{x})$ is Gaussian. (Also dim y $<$ dim x.) \item I(y,x) = H(y) + H(x) - H(x,y). Here, H(x) = H(x,y) so I(x,y) = H(y).  \item $H(z)=-\int dz p(z) \ln p(z) = \frac{1}{2}\ln|\Sigma|+\frac{D}{2}(1+\ln 2\pi )$.  \item I.e. want to choose $y$ with largest det of covariance matrix $|\Sigma|$, i.e. largest volume.  \item $\Sigma_y = A\Sigma_x A^T = AUW_XU^TA^T$ \item So $A$ should be aligned with the columns of $U$ which are associated with the largest eigenvalues (variances).  \item Projection to the principal component subspace preserves the most information about the data.  \end{itemize}

Mutual information I(X,Y) in terms of entropy; \begin{itemize} \item $H(X)+H(Y)-H(X,Y)$ \item $=H(X)-H(X|Y)$ \item $= H(Y) - H(Y|X)$ (MI is symmetric) \item $=H(X,Y)-H(X|Y)-H(Y|X)$ \end{itemize}

Entropy of a Gaussian $h(z)$; $h(z)=-\int dz p(z) \ln p(z) = \frac{1}{2}\ln|\Sigma|+\frac{D}{2}(1+\ln 2\pi )$

PC score; project vector onto first few (two e.g.) PCs.

How do you interpret a matrix? e.g. $U = [u_1 u_2 ...]$, and you have the product $Ux$.; Each $u_i$ is an axis (or directional vector), and each $x_i$ is how far you want to walk in the direction of $u_i$. \newline Note: this is not in the notes.

Briefly describe linear autoencoders.; \begin{itemize} \item Learn $\arg\min_{P,Q}||\hat{\mathbf{x}}-\mathbf{x}||^2, \mathbf{\hat{x}=Qy}$ (decoder 'generation'), $\mathbf{y=Px}$ (encoder 'recognition').  \item At the optimum, P and Q perform the projection and reconstruction steps of PCA (Baldi and Hornik 1989).  \end{itemize}

Derivative of a matrix inverse $\frac{d\mathbf{Y^{-1}}}{dx}$; $-Y^{-1}\frac{\partial{Y}}{dx}Y^{-1}$

Rank-Nullity theorem; rank(S) + Nullity(S) = dim(V), where $S:V\to W$. i.e. V is number of columns of matrix. k

Derivative of trace of matrix inverse multiplied by other matrices $\frac{d Tr \mathbf{(AX^{-1}B)^{-1}}}{dX}$; $-(X^{-1}BAX^{-1})^T$

Properties of an empirical covariance matrix; \begin{itemize} \item DxD, where D is the dimension of the data $x$ \item Symmetric \item Positive semi-definite \end{itemize}

Log likelihood of PPCA; $l = -\frac{N}{2}\log|2\pi C|-\frac{N}{2}Tr[C^{-1}S]$, where $C=\Lambda\Lambda^T+\psi I$.

Maximum likelihood expression for PPCA; Deriv wrt $\Lambda$: \begin{itemize} \item $\frac{\partial l}{\partial \Lambda} = N(-C^{-1}\Lambda+C^{-1}SC^{-1}\Lambda) = 0$ \item (Use trace inverse trick.) \item So at stationary point we have $\Lambda = SC^{-1}\Lambda$.  \item working: $l = -\frac{N}{2}\log|2\pi C|-\frac{N}{2}Tr[C^{-1}S]$, where $C=\Lambda\Lambda^T+\psi I$ \item and use trace inverse formula $\frac{d Tr \mathbf{(AX^{-1}B)^{-1}}}{dX}] = -(X^{-1}BAX^{-1})^T$ \end{itemize}

Singular Value Decomposition; SVD decomposes any n x p matrix $\mathbf{X}$.\begin{itemize} \item $X = USV^T$ \item dim U = nxn, $U^TU=I_n$.  \item dim V = pxp, $V^TV=I_p$ \item dim S = nxp, S diagonal.  \item Singular values are diagonal entries $[S]_{jj}$, all positive, ordered in decreasing order from upper left diagonal element.  \end{itemize} (From Intro to DL Background Maths slides)

ML learning of $\Lambda$ for PPCA: what to do after obtaining stationary point expression $\Lambda = SC^{-1}\Lambda, C=(\Lambda\Lambda^T+\psi I)$; \begin{itemize} \item Either (1) $\Lambda=0$, which turns out to be a minimum, or \item (2) $C=S \Rightarrow \Lambda\Lambda^T=S-\psi I$ \begin{itemize} \item Method 1: $rank(\Lambda\Lambda^T)\leq K\Rightarrow rank(S-\psi I) \leq K$ \item $\Rightarrow S$ has $D-K$ eigenvalues $=\psi$ and $\Lambda$ aligns with the space of remaining eigenvectors.  \item Method 2: Take the SVD $\Lambda = ULV^T$ on stat pt expression, use trick $U(L^2+\psi I)=(UL^2U^T+\psi I)U$ and take inverses of both sides to get $SU=U(L^2+\psi I), (L^2+\psi I)$ diagonal.  \item So columns of U are eigenvectors of S with eigenvalues given by $l^2_i + \psi$.  \item Thus, $\Lambda = ULV^T$ spans a space defined by K eigenvectors of S, \item and the lengths of the column vectors of L are given by the eigenvalues $-\psi$ (V selects an arbitrary basis in the latent space).  \end{itemize} \item Remains to show that the global ML solution is attained when $\Lambda$ aligns with the K leading eigenvectors. (claims this is intuitively reasonable.) \end{itemize}

PCA projection; $\Lambda\Lambda^Tx$. Note that the pseudo-inverse (check this is accurate) of $\Lambda$ is $\Lambda^T$.

Can we project $x_n \rightarrow \hat{x_n}$ trivially in PCA and in PPCA?; \begin{itemize} \item PCA yes since 'noise' is orthogonal to subspace.  \item PPCA no since noise is equal in all directions.  \end{itemize} 

Finding the projection for PPCA; \begin{itemize} \item Find the expected value $\mathbf{\bar{y}}_n=E[\mathbf{y_n|x_n}]$ and then take $\hat{x}_n=\Lambda\bar{y}_n$.  \item Find expected value by taking $p(y_n, x_n|\theta)$ and considering $x_n$ to be fixed.  \item Then get $p(y_n, x_n) \sim N(\mu, \Sigma)$, where $Sigma = (I+\Lambda^T\Psi^{-1}\Lambda)^{-1}=I-\beta\Lambda$ and $\mu=\Sigma\Lambda^T\Psi^{-1}x_n=\beta x_n$, where $\beta = \Sigma\Lambda^T\Psi^{-1}$.  \item Above: use $(I+P)^{-1}=I-(I+P)^{-1}P$.  \item Thus, $\hat{x}_n=\Lambda\mu=x_n - \Psi(\Lambda\Lambda^T+\Psi)^{-1}x_n$.  \item Above: use $(I+PQ)^{-1}P=P(I+QP)^{-1}$, and then add I and take away I, $I=\Psi\Psi^{-1}$.  \item Not the same projection as PCA. PPCA takes into account noise in the principal subspace.  \item As $\psi\rightarrow 0$, PPCA est approaches PCA value.  \end{itemize}

Interpret the results for projection of PPCA; \begin{itemize} \item Two sources of information about same variance \item Sum prior and likelihood weighted be precision of each \item FYI: $\Sigma = (I+\Lambda^T\Psi^{-1}\Lambda)^{-1}=I-\beta\Lambda$ and $\mu=\Sigma\Lambda^T\Psi^{-1}x_n=\beta x_n$, where $\beta = \Sigma\Lambda^T\Psi^{-1}$.	\end{itemize}

Name two models with different choices of noise that have an interesting difference in results; PPCA vs Factor Analysis. \newline Interesting bc we often spend lots of time on the generative model vs the noise model.

Difference between Factor Analysis and PPCA; \begin{itemize} \item Noise is not the same in each dimension: $\epsilon_d$ are indep $N(0,\Psi_{dd})$ gaussian noise.  \item i.e. if dims not equivalent, equal variance assumption is inappropriate.  \end{itemize}

Briefly describe dimensionality reduction; \begin{itemize} \item Find a low-dimensional projection of high-dimensional data that captures the correlation structure of the data.  \end{itemize}

Number of parameters for factor analysis; $DK + D - \frac{K(K+1)}{2}$. \begin{itemize} \item Correction $- \frac{K(K+1)}{2}$ is for subtracting extra DoF in model that don't appear in the likelihood.  \item Note if number of parameters $> \frac{D(D+1)}{2}$, model is not identifiable (even after accounting for rotational degeneracy discussed later) \item figure is from num of params in empirical covariance matrix \item Here, 'identifiable' means you can't separate the variance in the latent variable from the variance in the noise.  \item TODO: proof for above? where did the number come from? what is the rotational degeneracy?  \end{itemize}

What does it mean for a model to be identifiable?; As number of datapoints tends to infinity, should be able to uniquely identify $\theta$.

Is there a closed form solution for ML parameters for factor analysis?; No

Factor analysis projections compared to PPCA; Analysis from PPCA still applies, but $\Psi$ only diagonal but not spherical. i.e. \begin{itemize} \item $\hat{x}_n=\Lambda\mu=x_n - \Psi(\Lambda\Lambda^T+\Psi)^{-1}x_n$.  \item Note $\Lambda$ generally different from that found by PPCA.  \item and $\Lambda$ is not unique: latent space may be transformed by an arbitrary orthogonal transform without changing the likelihood. (also true for PPCA) \item FYI: $\Sigma = (I+\Lambda^T\Psi^{-1}\Lambda)^{-1}=I-\beta\Lambda$ and $\mu=\Sigma\Lambda^T\Psi^{-1}x_n=\beta x_n$, where $\beta = \Sigma\Lambda^T\Psi^{-1}$. \end{itemize}

Why is the $\Lambda$ in factor analysis generally different from that found in PPCA?; $\Psi$ affects estimate of $\Lambda$.

Why can the latent space for factor analysis be transformed by an arbitrary orthogonal transform without changing the likelihood?; Partially because the prior is preserved. (Poss also because noise is learned?)

Is the $\Lambda$ learned in FA or PPCA unique?; No. Latent space may be transformed by an arbitrary orthogonal transform without changing the likelihood.

How can you learn parameters for FA? (/issues associated with this); \begin{itemize} \item No spectral short-cut (closed-form solution) exists.  \item Likelihood can have more than one (local) optimum, making it difficult to find the global value.  \item For some data (`Heywood cases'), likelihood may grow unboundedly by taking one or more $\Psi_{dd}\rightarrow 0$.  \begin{itemize} \item Can eliminate problem by assuming prior on $\Psi$ with zero density at $\Psi_{dd}=0$, but results sensitive to precise choice of prior.  \end{itemize} \item can use e.g. gradient-based methods. Can also use EM, though EM does not solve issues mentioned.  \end{itemize}

Give an example of a prior that has zero density at $\Upsilon_{dd}=0$.; $\chi^2$ 

What does it mean for $\Psi_{dd}\rightarrow 0$? Is it realistic?; \begin{itemize} \item Appears that all variance in one dimension can be explained by model $\Lambda y$, so no need for noise in that dim.  \item Don't usually see this in real data.  \end{itemize}

Comparing FA, PPCA and PCA; \begin{itemize} \item PCA and PPCA rotationally invariant/covariant (since noise model has equal var in all dirs), FA not \begin{itemize} \item if $x\rightarrow Ux$ for unitary U, then $\lambda^{PCA}_{i}\rightarrow U \lambda^{PCA}_{(i)}$.  \end{itemize} \item FA measurement scale invariant, PCA and PPCA not (since expect noise to be different scale if measurement scale diff, but FA estimates noise level) \begin{itemize} \item if $x\rightarrow Sx$ for diagonal S, then $\Lambda^{FA}_{i}\rightarrow S \Lambda^{FA}_{(i)}$.  \item Pre-processing data to have equal variance on each axis does not eliminate scale-dependence of (P)PCA, since it assumes equal noise variance. Total variance has contributions from both $\Lambda\Lambda^T$ and noise.  \end{itemize} \item FA and PPCA define a probabilistic model, PCA does not \end{itemize}

Does pre-processing data to have equal variance in each dim eliminate scale-dependence of (P)PCA? ;\begin{itemize} \item No, since (P)PCA assumes equal noise variance. Total variance has contributions from both $\Lambda\Lambda^T$ and noise.  \item If originally equal noise (e.g. measured by same sensor), z-scoring destroys equal noise.  \end{itemize}

Classical Canonical Correlations Analysis; \begin{itemize} \item Given data vector pairs $\mathcal{D}=\{(\mathbf{u_1, v_1}), \mathbf{(u_2, v_2)},...\}$ in spaces $\mathcal{U}$ and $\mathcal{V}$ \item Find unit vectors $\upsilon_1\in\mathcal{U}, \phi_1\in\mathcal{V}$ s.t. the correlation of $\mathbf{u_i}^T\upsilon_1$ and $\mathbf{v_i}^T\phi_1$ is maximised.  \item As with PCA, repeat in orthogonal subspaces.  \end{itemize}

Probabilistic Canonical Correlations Analysis; \begin{itemize} \item Generative model with latent $y_i\in R^k$ \item $y\sim N(0,I)$ (like correlation between locations) \item $u\sim N(\Upsilon y, \Psi_u), \Psi_u \succeq 0$ \item $v\sim N(\Phi y, \Psi_v), \Psi_v \succeq 0$ (pos semidef) \item $\Psi$ are like local correlations e.g. between temperature and rainfall \item Block diagonal noise \end{itemize}

Limitations of Gaussian, FA and PCA models; \begin{itemize} \item Class of densities which can be modelled is too restrictive: \item only take into account mean and variance of the data.  \item (Use mixtures of distributions to expand the class of densities which can be modelled greatly.) \end{itemize}

When might mixture distributions arise naturally?; \begin{itemize} \item Clustering: Obs collected from different sources of data or different phases of a source. (Structure learning) \item Approximating densities: each component an element of a basis (Density estimation) \end{itemize}

Is $s_i\sim_{iid}Discrete[\pi]$ a multinomial distribution?; No, it is a categorical dist. \begin{itemize} \item Multinomial = dist on counts for many draws.  \item Here a single choice.  \end{itemize}

What is a Gaussian scale mixture model? (intuitively); \begin{itemize} \item Combining Gaussians with different variances, e.g. all mean 0.  \item Often for flexible prior e.g. for count values(?) \item Want underlying to be tractable but also to have heavier tails.  \end{itemize}

An example of a class of methods we approach as the number of components in a mixture model approaches infinity; Bayesian nonparametrics

MoG responsibility (no need to expand component density); $P(s_i=m|x_i) = \frac{P(x|s_i=m)\pi_m}{\sum_m P(x|s_i=m)\pi_m}$

For $P(\theta, .), \frac{\partial P}{\partial \theta} =$; $P\times \frac{\partial\log P}{\partial \theta}$.

MoG log likelihood; $\log P(\{\mathbf{x}\}|\{\mathbf{\mu_m} \},\{\mathbf{\Sigma_m} \}) = \sum_{i=1}^N\log \sum_{m=1}^k \frac{\pi_m}{\sqrt{|2\pi\Sigma_m|}}\exp(-\frac{1}{2}(x_i-\mu_m)^T\Sigma^{-1}_m(x_i-\mu_m)$ \newline Note the likelihood does not lie in the exponential family.

Log likelihood for a mixture model; $l=\sum_{i=1}^N\log \sum_{m=1}^k \pi_mP_m(x_i, \theta_m)$

Partial derivative $\frac{\partial \mathcal{L}}{\partial \theta_m}$ for a mixture model; $\sum_{i=1}^N r_{im} \frac{\partial\log P_m(x_i, \theta_m)}{\partial\theta_m}$ using $\frac{\partial P}{\partial \theta} = P \times \frac{\partial\log P}{\partial \theta}$

Partial derivative $\frac{\partial \mathcal{L}}{\partial \pi_m}$ for a mixture model; $\sum_{i=1}^N \frac{r_{igit m}}{\pi_m}-\lambda = 0$, $\pi_m = \frac{1}{N}\sum_{i=1}^N r_{im}$.

MoG derivatives: $\frac{\partial \mathcal{L}}{\partial \mathbf{\mu_m}} =$; $\sum_{i=1}^n r_{im}\Sigma^{-1}_m(x_i-\mu_m)$

%MoG derivatives: $\frac{\partial \mathcal{L}}{\partial \mathbf{\Sigma^{-1}_m}} =$; $\frac{1}{2}\sum_{i=1}^n r_{im}(\Sigma_m-(x_i-\mu_m)(x_i-\mu_m)^T)$. \newline Remember to use Lagrange multiplier for $\pi_m$ because prob mass needs to sum to 1.

Describe K means (relate it to another model, comment on convergence); \begin{itemize} \item Limiting case of the mixture of Gaussians: $\pi_m = 1/k, \Sigma_m=\sigma^2I, \sigma^2\rightarrow 0$.  \item Then the responsibilities become binary: $r_{im}\rightarrow\delta(m, \arg\min_m ||x_i-\mu_m||^2)$.  \item i.e. 1 for component with closest mean, 0 for all other components. $->$ Solve directly for means by setting gradient to 0.  \item Iterate: \begin{itemize} \item Assign each point to closest mean (assign $r_{im}=$limit.) \item Update means to average of assigned points (set $\mathbf{\mu_m}=\frac{\sum_i r_{im}\mathbf{x_i}}{\sum_i r_{im}})$ \end{itemize} \item Usually converges within a few iterations, although fixed point depends on initial values chosen for $\mu_m$.  \item No learning rate, but assumptions limiting.  \end{itemize}

Problems with iterative gradient descent algorithms; \begin{itemize} \item Slow convergence for gradient based methods \item Gradient based methods may develop invalid (not pd) covariance matrices (e.g. if step size large. Need to put bounds, make constrained opt.) \item Local minima: end config may depend on starting state.  \item How adjust num clusters k? Likelihood alone no good.  \item Singularities: components with a single data point will have covar going to zero, $l\rightarrow\inf$. (important failure mode for MoG with $\geq 2$ Gaussians. Usually if init with large covar, don't see degeneracy.) \end{itemize}

\end{document}
